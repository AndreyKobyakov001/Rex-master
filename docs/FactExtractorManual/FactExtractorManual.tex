% Style Sheet for Document.
% Adapted from Overleaf template.
%https://www.overleaf.com/latex/templates/assessment-report/hgqqnkgvsmty#.WhIQ-nUrKV5
\documentclass[11pt,a4paper,titlepage]{article}
\usepackage[utf8]{inputenc}
\usepackage[margin=2cm,headheight=13.6cm]{geometry}
\usepackage{color}
\usepackage{listings}
\usepackage{graphicx}
\usepackage{caption}
\usepackage{float}
\usepackage{titling}
\usepackage{pgfkeys}
\usepackage[hidelinks]{hyperref}
\usepackage[inline,shortlabels]{enumitem}
\usepackage[inline]{enumitem}
\definecolor{mygreen}{RGB}{0,127,0}
\definecolor{mygray}{RGB}{100,100,100}
\definecolor{mymauve}{RGB}{100,32,255}
\definecolor{lgray}{RGB}{230,230,230}
\lstset{ %
  frame=none,
  backgroundcolor=\color{white},   % choose the background color; you must add \usepackage{color} or \usepackage{xcolor}
  basicstyle=\footnotesize\ttfamily,        % the size of the fonts that are used for the code
  breakatwhitespace=false,         % sets if automatic breaks should only happen at whitespace
  breaklines=true,                 % sets automatic line breaking
  captionpos=t,                    % sets the caption-position to bottom
  commentstyle=\color{mygreen},    % comment style
  deletekeywords={...},            % if you want to delete keywords from the given language
  escapeinside={\%*}{*)},          % if you want to add LaTeX within your code
  extendedchars=true,              % lets you use non-ASCII characters; for 8-bits encodings only, does not work with UTF-8
%  frame=single,                    % adds a frame around the code
  keepspaces=true,                 % keeps spaces in text, useful for keeping indentation of code (possibly needs columns=flexible)
  keywordstyle=\color{blue},       % keyword style
  language=,                 % the language of the code
  morekeywords={*,...},            % if you want to add more keywords to the set
  numbers=left,                    % where to put the line-numbers; possible values are (none, left, right)
  numbersep=5pt,                   % how far the line-numbers are from the code
  numberstyle=\tiny\color{mygray}, % the style that is used for the line-numbers
  rulecolor=\color{black},         % if not set, the frame-color may be changed on line-breaks within not-black text (e.g. comments (green here))
  showspaces=false,                % show spaces everywhere adding particular underscores; it overrides 'showstringspaces'
  showstringspaces=false,          % underline spaces within strings only
  showtabs=false,                  % show tabs within strings adding particular underscores
  stepnumber=1,                    % the step between two line-numbers. If it's 1, each line will be numbered
  stringstyle=\color{mymauve},     % string literal style
  tabsize=4,                       % sets default tabsize to 2 spaces
  aboveskip=3mm,
  belowskip=3mm,
}

\usepackage{fancyhdr}
\pagestyle{fancy}
\rhead{Muscedere, Atlee, Godfrey, and Davis}
\usepackage{datetime}
\usepackage{moresize}

\newdateformat{monthyeardate}{%
  \monthname[\THEMONTH] \THEYEAR}

\newcommand{\rulebreak}{%
	\par%
	\vspace{0.9cm}%
    \noindent\rule{4cm}{0.4pt}%
    \vspace{1.2cm}%
    \par%
}

\newcommand{\coverpage}[1]{%
	\pagenumbering{roman}%
	\thispagestyle{empty}%
	\lhead{\textsc{#1}}%
    \title{#1}%
    \author{Bryan J Muscedere\\Professor Joanne Atlee\\Professor Michael Godfrey\\Dr. Ian Davis\\\bigskip Updated by: Sunjay Varma (April 2019)}%
    \newgeometry{left=5cm,bottom=2cm,right=5cm,top=2cm}%
	\begin{center}\hspace{0pt}\vspace{-1cm}\vfill%
    \uppercase{Cheriton School of Computer Science\\
    University of Waterloo}
	\rulebreak%

    \vspace{0.5cm}
    {\Huge\textbf{\textit{#1}}}

    \vspace{0.5cm}
	\theauthor%
	\par%
	\vspace{0.9cm}%
    \noindent\rule{4cm}{0.4pt}%
    \vspace{0.45cm}
    \tableofcontents%
	\rulebreak%
    \monthyeardate\today\par
    \hspace{0pt}
	\end{center}%
    \vfill
    \hspace{0pt}
	\pagebreak%
    \restoregeometry%
    \pagenumbering{arabic}%
}

% Custom arguments for /fig command
\pgfkeys{
 /fig/.is family, /fig,
 default/.style =
  {scale = 1,
   angle = 0},
 scale/.estore in = \figScale,
 angle/.estore in = \figAngle
}
\newcommand{\fig}[2][]{%
	\pgfkeys{/fig, default, #1}%
	\begin{figure}[H]%
    \centering
    \includegraphics[angle=\figAngle,width=\figScale\textwidth]{#2}%
	\end{figure}%
}

\newcommand{\filename}[1]{%
	\texttt{#1}%
}

\newcommand{\vhdl}[1]{%
  \lstinputlisting[language=vhdl]{#1}
}

\newcommand*\paths[1]{\lstset{inputpath=#1}\graphicspath{#1}}

\definecolor{codegreen}{rgb}{0,0.6,0}
\definecolor{codegray}{rgb}{0.5,0.5,0.5}
\definecolor{codepurple}{rgb}{0.58,0,0.82}
\definecolor{backcolour}{rgb}{0.95,0.95,0.92}
\lstdefinestyle{GrokStyle}{
    backgroundcolor=\color{backcolour},
    commentstyle=\color{codegreen},
    keywordstyle=\color{magenta},
    numberstyle=\tiny\color{codegray},
    stringstyle=\color{codepurple},
    basicstyle=\footnotesize,
    breakatwhitespace=false,
    breaklines=true,
    captionpos=b,
    keepspaces=true,
    numbers=left,
    numbersep=5pt,
    showspaces=false,
    showstringspaces=false,
    showtabs=false,
    tabsize=2
}


\begin{document}

\coverpage{Fact Extractor Manual}

\section{Installing Rex}

Rex is a C++ fact extractor that utilizes the clang AST to generate "facts" from
C++ source files. This section will provide information that installs the
required libraries and tools used to install the Rex extractor. Before Rex can
be built, it requires the following to be installed on the target system:
\textbf{CMake} 3.0.0 or greater, \textbf{Boost} 1.69 or greater,
\textbf{LibSSL}, and \textbf{Clang} 8.0.0. CMake is used to build Rex, Boost is
a collection of C++ libraries that is used to process command line arguments and
directory information, and Clang is used to obtain AST information about the
source code currently being processed. The Clang API provides methods for
operating on C++ language features and carries out the brunt of the C++
analysis.

The remainder of this section provides information on how to install these
required libraries and how to build Rex from source. These instructions are for
Linux-based systems. Section \ref{sec:CRPre} describes how to install CMake,
Boost, and Clang. If any of these are already installed on the target system, it
can be skipped. Section \ref{sec:RexBuild} describes how to build Rex from
source.

If you find any errors in this document or if the steps do not work, please
report what you find to someone working on the project. If you yourself work on
the project, please fix whatever you find. The source code for these
instructions can be found in the \texttt{docs/} directory of the Rex repository.

\subsection{Prerequisites}
\label{sec:CRPre}
\paragraph{CMake}

If using a Ubuntu or Debian-based system, installing CMake is as simple as using
\texttt{apt-get}. This is done using the following two commands:

\begin{lstlisting}[language=bash]
  	$ sudo apt-get install cmake
	$ cmake --version
\end{lstlisting}

\noindent Using \texttt{apt-get} might not install the latest version of CMake.
However, as long as it installs a version of CMake greater than \texttt{3.0.0},
it can be used to build ClangEx and Rex. If this is the case, proceed to the
Boost installation instructions.

If this method did not work, CMake must be built from source. To do this, the
latest version of CMake source needs to be downloaded from the CMake website,
compiled, and then installed. This guide provides instructions on how to build
CMake \texttt{3.7.0} from source. The first step is to download the CMake source
code and unzip it. This can be from the command line using the following
commands:

\begin{lstlisting}[language=bash]
	$ wget https://cmake.org/files/v3.7/cmake-3.7.0.tar.gz
	$ tar xvzf cmake-3.7.0.tar.gz
	$ cd cmake-3.7.0
\end{lstlisting}

\noindent Once in the \texttt{cmake-3.7.0} directory, CMake can be configured
and install on the target system. This process may take several minutes. To do
this, use the following commands:

\begin{lstlisting}[language=bash]
	$ ./configure
	$ make
	$ make install
	$ cmake --version
\end{lstlisting}

\noindent If these steps completed successfully, CMake is now installed and
ready for use.

\paragraph{Boost}

Do \textbf{not} install Boost using the \texttt{apt-get} package manager. The
version of Boost provided is not guaranteed to be up to date and can otherwise
cause you many problems. Instead, visit the Boost downloads website and download
the Boost 1.69 source code.

\bigskip
Boost Downloads Page: \url{https://www.boost.org/users/download/}
\bigskip

\noindent Then, extract the files into a \texttt{boost\_1\_69\_0} directory and
run the following commands:

\begin{lstlisting}[language=bash]
	$ cd path/to/boost_1_69_0
	$ ./bootstrap.sh
	$ ./b2
\end{lstlisting}

\noindent This will build Boost 1.69 in that directory, and otherwise leave your
system untouched.

\paragraph{Clang}

Although Clang exists as a package that can be installed using \texttt{apt-get}, it needs
to be installed from source to take advantage of Clang’s API. To do this,
source code needs to be checked out from the official Clang repository, compiled,
and then installed. The first step is to checkout the Clang source code.
This can be done by executing the following:

\begin{lstlisting}[language=bash]
$ wget releases.llvm.org/8.0.0/llvm-8.0.0.src.tar.xz
$ tar xf llvm-8.0.0.src.tar.xz
$ wget releases.llvm.org/8.0.0/cfe-8.0.0.src.tar.xz
$ tar xf cfe-8.0.0.src.tar.xz && mv cfe-8.0.0.src llvm-8.0.0.src/tools/clang
$ wget releases.llvm.org/8.0.0/clang-tools-extra-8.0.0.src.tar.xz
$ tar xf clang-tools-extra-8.0.0.src.tar.xz 
$ mv clang-tools-extra-8.0.0.src llvm-8.0.0.src/tools/clang/tools/extra
\end{lstlisting}

These commands will download the LLVM and Clang source code to a directory called
llvm-8.0.0.src in the current working directory. Clang can now be built. This process
can take up to several hours and uses a large amount of RAM and disk space (make
sure you have enough swap space). This guide shows how to build Clang in a
directory called Clang-Build that is adjacent to the llvm-8.0.0.src directory. To use
another directory, simply replace the Clang-Build string in the following
commands with another directory name.

The following commands build Clang in the Clang-Build directory:

\begin{lstlisting}[language=bash]
$ mkdir Clang-Build
$ cd Clang-Build
$ cmake -G "Unix Makefiles" -DCMAKE_BUILD_TYPE=Release -DLLVM_ENABLE_EH=ON -DLLVM_ENABLE_RTTI=ON ../llvm-8.0.0.src
$ make
$ sudo make install
\end{lstlisting}

Once these steps have been completed, Clang and LLVM have been installed.
By typing clang --version, Clang should report that it is version 8.0.

\subsection{Building Rex}
\label{sec:RexBuild}

To build Rex, the source code needs to be checked out from the Rex git
repository and then built using CMake. If Rex fails to build, first ensure that
all tools and libraries installed in the previous section are present on the
target system.

To download Rex, you first need to checkout the source code. Run the \texttt{git
clone} command to download the repository into a directory called \texttt{Rex}.
Before checking out the repository, ensure that you are in a directory where you
want to install Rex.

\bigskip

\noindent Next, before Rex can be built, three environment variables must be
set:

\begin{itemize}
  \item \texttt{LLVM\_PATH} should be set to the \texttt{clang\_llvm\_8.0.0} directory
  \item \texttt{CLANG\_VER} should be set to the string \texttt{8.0.0}
  \item \texttt{BOOST\_ROOT} should be set to the \texttt{boost\_1\_69\_0} directory
\end{itemize}

\noindent To set these variables, open up \texttt{.bashrc} located in the home
directory and add the following lines to the bottom of the file:

\begin{lstlisting}[language=bash]
	$ export LLVM_PATH=<PATH_TO_clang_llvm_8.0.0>
	$ export CLANG_VER=8.0.0
	$ export BOOST_ROOT=<PATH_TO_boost_1_69_0>
\end{lstlisting}

\noindent Restart the terminal to ensure these variables have been exported.

\bigskip

\noindent The next step is to build Rex. These steps install Rex in a directory
called \texttt{Rex-Build} adjacent to the \texttt{Rex} directory. To install it
somewhere else, simply replace the \texttt{Rex-Build} string in the following
commands with a desired directory name. The following steps install Rex in the
\texttt{Rex-Build} directory:

\begin{lstlisting}[language=bash]
	$ cd Rex-Build
	$ cmake -G "Unix Makefiles" ../Rex
	$ make -j 8
\end{lstlisting}

\noindent By running these commands, CMake will build the source code and an
executable called \textit{Rex} will be created inside the \texttt{Rex-Build}
directory. To verify that Rex built correctly, run:

\begin{lstlisting}[language=bash]
	$ ./Rex --help
\end{lstlisting}

\noindent If Rex built, this command will run the Rex executable and print out
some help information about the kinds of arguments you can pass to Rex. Now, Rex
is ready to process C++ source files and generate tuple-attribute models!

\section{Using Rex}

Rex is a command line tool. This section describes the various ways you can
configure it using arguments passed to the \texttt{Rex} executable built in the
previous section.

\subsection{Separate Compilation}

To run Rex, pass in one or more source files or directories that you would like
it to analyze. For directories, Rex will recursively search for both CPP source
files and header files. All common C++ file extensions are supported. Here's an
example:

\begin{lstlisting}[language=bash]
	$ ./Rex foo.cpp foo2.cpp bar/
\end{lstlisting}

\noindent Rex will run clang on both \texttt{foo.cpp} and \texttt{foo2.cpp}. It
will also walk the \texttt{bar} directory recursively looking for both source
files and header files. All files found will be analyzed as well.

\bigskip

\noindent To run multiple analyses in parallel, pass in the \texttt{--jobs} flag
(abbreviated \texttt{-j}) as shown below:

\begin{lstlisting}[language=bash]
	$ ./Rex foo.cpp foo2.cpp bar/ -j 8
\end{lstlisting}

\noindent This will usually result in a faster analysis because Rex can analyse
multiple source files at once.

\bigskip

By default, Rex will analyse these files, but not attempt to link together the
results in any way. This enables us to do separate compilation and optimize
re-running Rex when only a few files have changed. The results of the analysis
of each file are stored in ``TA Object Files''. You can find those files in the
directory where you ran Rex. Each file will have the \texttt{.tao} file
extension. This is a custom, Rex-specific file format that is specially designed
to make the linking process fast.

\bigskip

One caveat to be aware of: Most files require at least some compiler flags in
order for clang to be able to compile them. The way these flags are discovered
is with a special file called a compilation database. If the project being
processed uses any compiler flags, a compilation database will need to be
generated for that project and placed in the root directory of all the project
source files. All files being analyzed by Rex must either be in the same
directory as the database or in some descendant directory. A compilation
database is a JSON file that has an entry for \textbf{each} file that Rex will
analyze. If a file is not present in that database, Rex may fail to analyze that
file at all. If a compilation database does not exist, Rex will show a warning
and attempt to proceed without it.

\subsection{Linking}

To produce a TA file from the generated \texttt{.tao} files, pass the
\texttt{--output} flag (abbreviated \texttt{-o}) as shown below:

\begin{lstlisting}[language=bash]
	$ ./Rex foo.cpp foo2.cpp bar/ -j 8 -o bar.ta
\end{lstlisting}

\noindent This will analyze each file, generate \texttt{.tao} files for each,
and link them together to produce a TA file with the filename \texttt{bar.ta}.
If you have already generated  the \texttt{.tao} files and do not want to re-run
the analysis, you can pass those directly as well:

\begin{lstlisting}[language=bash]
	$ ./Rex *.tao -o bar.ta
\end{lstlisting}

\noindent Rex even supports a mix of source file/directory arguments and
\texttt{.tao} file arguments:

\begin{lstlisting}[language=bash]
	$ ./Rex foo.cpp.tao foo2.cpp bar/ -j 8 -o bar.ta
\end{lstlisting}

\noindent In this case, the \texttt{.tao} file arguments will only be used
during linking and the other provided files/directories will be analyzed and
linked as would be normally expected.

You can mix and match these styles of passing arguments to optimize your run of
Rex so that it only ever analyses the source files you want it to. This style of
command line arguments is similar to what you may be used to with most
compilers.

\subsection{Including Files in the Analysis}

Rex will not analyze any files that you do not provide as arguments. That means
that if you are passing individual files into Rex, you will often need to
provide \textbf{both} the CPP source file and the associated header file in
order to get any results:

\begin{lstlisting}[language=bash]
	$ ./Rex foo.cpp foo.h -j 8 -o bar.ta
\end{lstlisting}

\noindent If you omit the header file, you may find that the generated TA file
is empty or far smaller than you expected. The reason for this is because most
declarations are in header files and Rex needs to analyze those header files in
order to be able to link their declarations.

If you use a directory instead, Rex will take care of picking up both types of
files automatically. That is usually the preferred method of using Rex unless
you are trying to diagnose a specific problem or otherwise need to narrow down
the analysis to specific files.

It's not that the source file and header file need to be run in the same
command. The only requirement is that when linking into a TA file, the
\texttt{.tao} file for both the source and the header are included.

\begin{lstlisting}[language=bash]
	$ ./Rex foo.cpp
	$ ./Rex foo.h
	$ ./Rex foo.cpp.tao foo.h.tao -o bar.ta
\end{lstlisting}

\noindent A great consequence of this slightly unconventional setup is that only
the files we intend to be included are included in our analysis. That means that
system files are completely ignored unless we explicitly pass their
files/directories into Rex.  Automatically detecting the header files associated
with a given source file is non-trivial in C++. This solution gives us
flexibility without too much added complexity.

\end{document}
